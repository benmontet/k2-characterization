% The rules:
% I guess same as DFM because he'll complain if we don't use the 
% character limit.
% New sentences on each line.
% Citation format is \citet{AuthornameYY}.
% References go into papers.bib alphabetically, with ties broken
% chronologically.

%\documentclass[]{aastex}
\documentclass{emulateapj}
%\documentclass[]{emulateapj}

\bibliographystyle{apj}
\usepackage{subfigure}
\usepackage{url}
\usepackage{hyperref}
\usepackage{datetime}
\usepackage{longtable}
\usepackage{natbib}
\usepackage{amsmath}
\usepackage{ulem}
\usepackage{bm}
\usepackage[usenames,dvipsnames]{xcolor}

%\usepackage{pdflscape}

% FIGSET-MACROS-BEGIN
\newcommand{\noprint}[1]{}
\newcommand{\figsetstart}{{\bf Fig. Set} }
\newcommand{\figsetend}{}
\newcommand{\figsetgrpstart}{}
\newcommand{\figsetgrpend}{}
\newcommand{\figsetnum}[1]{{\bf #1.}}
\newcommand{\figsettitle}[1]{ {\bf #1} }
\newcommand{\figsetgrpnum}[1]{\noprint{#1}}
\newcommand{\figsetgrptitle}[1]{\noprint{#1}}
\newcommand{\figsetplot}[1]{\noprint{#1}}
\newcommand{\figsetgrpnote}[1]{\noprint{#1}}
% FIGSET-MACROS-END
\usepackage{color}

\newcommand{\ron}{\color{red}} 
\newcommand{\bon}{\color{blue}} 
\newcommand{\gon}{\color{green}} 
\newcommand{\coff}{\color{black}\,}

\newcommand{\rprs}{{$R_p/R_{\star}$}}

\newcommand{\eg}{{\it e.g.}}
\newcommand{\ie}{{\it i.e.}}
\newcommand{\project}[1]{\textsl{#1}} 
\newcommand{\kep}{\project{Kepler}}
\newcommand{\KT}{\project{K2}}
\newcommand{\vsini}{{$V \sin i$}}
\newcommand\teff{\ensuremath{T_\text{eff}}}
\newcommand{\kms}{{km\,s$^{-1}$}}
\newcommand{\gcc}{{g\,cm$^{-3}$}}
\newcommand{\rstar}{{$R_\star$}}
\newcommand{\rhostar}{{$\rho_\star$}}
\newcommand{\mearth}{{M$_\oplus$}}
\newcommand{\rearth}{{R$_\oplus$}}
\newcommand{\rsun}{{R$_\odot$}}
\newcommand{\msun}{{M$_\odot$}}
\newcommand{\mjup}{{M$_\textrm{Jup}$}}
\newcommand{\rjup}{{R$_\textrm{Jup}$}}


\newcommand{\mstar}{{$M_\star$}}
\newcommand{\logg}{{log(g)}}
\newcommand{\mh}{{[M/H]}~}
\newcommand{\feh}{{[Fe/H]}~}
\newcommand{\afe}{{[$\alpha$/Fe]}~}

\newcommand{\paperip}{\citep{Foreman-Mackey15}}
\newcommand{\paperit}{\citet{Foreman-Mackey15}}

\newcommand{\is}{importance sampling}
\newcommand{\Is}{Importance sampling}
%\newcommand{\h2ok2}{{$\rm H_2O-K2$}}

\newcommand{\todo}[3]{{\color{#2} \emph{#1} TO DO: #3}}
\newcommand{\btmtodo}[1]{\todo{BEN}{red}{#1}}



\begin{document}
\title{(Working Title) Properties of the \KT\ Campaign 1 Planets, 
False Positive Probabilities, and Confirmation of $N$ Systems. You 
Were Warned: it's a Working Title.}

\author{
% The Heavy Lifters
Benjamin T. Montet\altaffilmark{1,2}, 
Daniel Foreman-Mackey\altaffilmark{3},
Timothy D. Morton,
% The Benefactors
John Asher Johnson,
David W. Hogg,
% The Observers, as PIs 
% (ranked by integration time x collecting area)
Brendan P. Bowler,
David W. Latham,
Andrew W. Mann,
% The observers, as a group
And Whoever DaveL Tells Us to Add as TRES Team Members
}

\email{btm@astro.caltech.edu}

\altaffiltext{1}{Cahill Center for Astronomy and Astrophysics, California Institute of Technology, 1200 E. California Blvd., MC 249-17, Pasadena, CA 91106, USA}
\altaffiltext{2}{Harvard-Smithsonian Center for Astrophysics, 60 Garden
Street, Cambridge, MA 02138, USA}
\altaffiltext{3}{Department of Astronomy, University of Michigan, Ann Arbor, MI 48109, USA}


\date{\today, \currenttime}

\begin{abstract}
\btmtodo{This is where the abstract will go}
Of the 36 planet candidates presented in \paperit{} we confirm 
\btmtodo{Nconfirm} as planets and \btmtodo{NFP} as false positives
at the \btmtodo{confidence} level. Of particular interest is EPIC 201912552, a bright (K=8.9) M2 dwarf
hosting a \btmtodo{Radius} planet with $T_{eq} = $\btmtodo{Temp} and an 
orbital period of 33 days.
\end{abstract}

\keywords{\btmtodo{Keywords}}

\maketitle

\section{Introduction}


\section{Stellar Properties}
\subsection{Photometry}

With the exception of one star in our sample (EPIC 201912552), we do not have
spectroscopic data with which to characterize the stellar properties. 
Instead, we rely on photometry.
For each system, we query the VizieR database of astronomical catalogues
\citep{Ochseinbein00}. 
We record the $B$, $V$, $g'$, $r'$, and $i'$ magnitudes and their
uncertainties from the AAVSO Photometric all-sky survey (APASS) DR6 
\btmtodo{Cite: Who?}, as reported in the UCAC4 Catalogue \citep{Zacharias12}.
We also record the $J$, $H$, and $K$ magnitudes and their uncertainties 
as found in the 2MASS All-Sky Catalog of Point Sources \citep{Cutri03}
and the four WISE magnitudes and uncertainties from the ALLWise Data
Release \citep{Cutri13}.
For all except two of our targets, the $W4$ band is only an upper limit.
These data are reported in Table \ref{Tab:Photometry}.

\btmtodo{Make this table!}

\subsection{Stellar Models}
To convert the observed photometric data into physical properties for each
star, we employ the use of stellar models. 
We first download the isochrone grids from the 2012 updated photometric
systems for the relevant bandpasses from the Dartmouth Stellar Evolution
Database \citep{Dotter08}. 
We then interpolate the isochrones onto a grid uniformly spaced in
stellar mass (from 0.20 to 2.0 \msun), metallicity (from \feh$=-1.5$ to 
\feh$=+0.5$), age (from 1.0 to 14.0 Gyr), and log distance (from 10 pc to 
1000 pc). 
In our interpolation, we assume \afe$=0.0$ and $Y=0.2741$, the solar value.
We then compare the expected apparent magnitudes to our observed photometry
to measure the likelihood of each grid point.
For each parameter, we then marginalize our likelihood over all other 
parameters in order to estimate the value of the physical properties
for each star.

Such a scheme enables us to infer the statistical uncertainties on the
mass, radius, and effective temperature.
However, we are subject to biases induced by systematics in the models themselves.
There is some evidence that the Dartmouth models may underpredict 
radii by $\sim 15\%$ when compared to other methods \citep{Newton15,
Montet15}. 
Such an effect may be the result of the Dartmouth model reliance on BT-Settl
atmospheres, which have been shown to predict near-IR colors that are too blue
\citep{Thompson14}.
We find that by measuring stellar parameters without any WISE data (that is, using
only $BVgriJHK$), we largely reproduce the same effective temperatures as when
the WISE data is included. 
However, at the low-mass end including the WISE data systematically increases the 
measured radius and increases \logg, as shown in \btmtodo{Figure}.
Suggestively, spectroscopic stellar parameters of EPIC 201912552 
(\textsection\ref{Spexobs}) suggest the derived photometric properties better 
represent the true physical parameters when the WISE data is included. 


\btmtodo{Show plot}



\subsection{SpeX Spectroscopy}
\label{Spexobs}


A near-infrared spectrum of EPIC 201912552 was obtained using the updated SpeX 
(uSpeX) spectrograph \citep{Rayner03} on the NASA Infrared Telescope Facility 
(IRTF) on January 29 2015 (UT). 
SpeX observations were taking using the short cross-dispersed mode and the
0.3$\times15\arcsec$ slit, which provides simultaneous coverage from 0.7 
to 2.5$\mu$m at $R\simeq2000$. 
The target was observed at two positions along the slit to subsequently subtract 
the sky background. Eight spectra were taking following this pattern, which provided 
a final S/N of $>150$ per resolving element. 
The spectrum was flat fielded, extracted, wavelength calibrated, and stacked 
using the \textit{SpeXTool} package \citep{Cushing04}. 
An A0V-type star was observed immediately after 
the target, which was used to create a telluric correction using the 
\textit{xtellcor} package \citep{Vacca03}.

An optical spectrum was obtained using the SuperNova Integral Field Spectrograph
\citep[SNIFS,][]{Aldering02,Lantz04} on the University of Hawai'i 
2.2m telescope on the night of January 30 2015. 
SNIFS provides simultaneous coverage from 3200\AA--9700\,AA\ at a resolution 
of $\simeq1000$. Final S/N of the spectrum was $>100$ pre resolving element 
in the red ($\sim6000$\AA). 
Details of the SNIFS reduction, including dark, bias, and flat-field corrections,
cleaning the data of bad pixels and cosmic rays, and extraction of the 
one-dimensional spectrum are described in \citet{Bacon01} and 
\citet{Aldering06}. 
Flux calibration was performed using a separate pipeline described in \citet{Mann15}. 

\teff\ was calculated by comparing our optical spectra with the CFIST
suite\footnote{http://phoenix.ens-lyon.fr/Grids/BT-Settl/CIFIST2011/} of the BT-SETTL
version of the PHOENIX atmosphere models \citep{Allard13}, which gave a temperature
of 3503 $\pm$ 60\,K. 
More details of this procedure are given in \citet{Mann14} and
\citet{Gaidos14}. 
This method was used because it is known to accurately reproduce empirical 
\teff\ values from long-baseline optical interferometry \citet{Boyajian12}. 

Metallcity was determined using the procedures from \citet{Mann13a}, in which the
authors provide empirical relations between atomic features and M dwarf
metallicity, calibrated using wide binaries. 
We adopted the weighted mean of the $H-$ and $K-$band calibrations, 
which yielded a metallicity of 0.09$\pm$0.09.

We combined the derived \teff\ and [Fe/H] values with the empirical 
\teff-[Fe/H]-$R_*$ relation from \citet{Mann15} to compute a radius. 
Accounting for measurement and calibration errors in [Fe/H] and \teff\ we calculated 
a radius 0.394$\pm0.038R_\odot$. 
We use these parameters instead of the derived photometric properties for this target.

\section{Planet Properties}

This is Dan?

\section{False Positive Analysis}
\subsection{Adaptive Optics Imaging}

This is Brendan.

\subsection{Known Background Stars}
The P3K AO system has a field of view of $3.84 \times 3.84$ arcseconds.
\btmtodo{Check with Brendan to make sure he used this mode}.
Each \KT\ pixel is a square of $3.98 \times 3.98$ arcseconds. 
A background eclipsing binary within a few \KT\ pixels of our target 
stars could mimic a transit signal while evading detection by P3K.
Such wide eclipsing binaries should appear in seeing-limited ground-based
surveys.

To investigate the possibility that such wide companions exist,
we query the ninth data release of the Sloan Digital Sky Survey 
\citep[SDSS DR9,][]{Ahn12}. 
For each target, we return
all stars within 12\arcsec with $\Delta r < 8$ relative to each object
of interest.
Of the 31 stars in our sample, four have such a companion.
Unlike the original \kep\ field, the field for \KT\ Campaign 1 is 
well out of the galactic plane, so the rate of giant, distanct background
stars is significantly lower.
We include a list of these potential contaminants as Table 
\ref{Table:BGstars}.

%Table:BGstars
% Primary    RA Dec Separation DeltaR         MaxDepth
% 201295312  &  &   &          & 7.10 \pm 0.10 & 0.75
% 201465501  &  &   &          & 7.90 \pm 0.25 & 0.36
% 201546283  &  &   &          & 5.87 \pm 0.06 & 2.31
% 201779067  &  &   &          & 4.21 \pm 0.03 & 10.56

In Table \ref{Table:BGstars}, the ``maximum depth'' column represents
the maximum observed ``transit'' depth if the transit were actually caused
by a total eclipse of the hypothetical background binary system, inducing
a 50\% flux decrement in the background star's apparent brightness...

\subsection{Archival Imaging}

This is Ben.


\subsection{TRES Radial Velocities}

\subsection{False Positive Rates (and confirmations?)}
This is Tim.

\section{Potentially Interesting Systems}
\subsection{The One Around the M dwarf}



\subsection{Some Multi (possible multi of these}

\subsection{Others? Really bright ones amenable to followup?}

\section{Results and Discussion}

Something about Multis and TTVs. 




\acknowledgements
We thank [People] for [things] which improved the quality of this manuscript.


We are grateful to the entire Kepler team, past and present. 
Their tireless efforts were all essential to the tremendous success of the mission and the successes of K2, present and future.


Some of the data presented in this paper were obtained from the Mikulski
Archive for Space Telescopes (MAST). 
STScI is operated by the Association of Universities for Research 
in Astronomy, Inc., under NASA contract NAS5--26555. 
for MAST for non--HST data is provided by the NASA Office of Space 
Science via grant NNX13AC07G and by other grants and contracts. 

This paper includes data collected by the \kep\ mission. 
Funding for the \kep\ mission is provided by the NASA Science 
Mission directorate.

This paper includes data collected by the Sloan Digital Sky Survey.
Funding for SDSS-III has been provided by the Alfred P. Sloan Foundation,
the Participating Institutions, the National Science Foundation, and the
U.S. Department of Energy Office of Science. The SDSS-III web site is
http://www.sdss3.org/.
SDSS-III is managed by the Astrophysical Research Consortium for the
Participating Institutions of the SDSS-III Collaboration including the
University of Arizona, the Brazilian Participation Group, Brookhaven
National Laboratory, Carnegie Mellon University, University of Florida,
the French Participation Group, the German Participation Group, Harvard
University, the Instituto de Astrofisica de Canarias, the Michigan
State/Notre Dame/JINA Participation Group, Johns Hopkins University,
Lawrence Berkeley National Laboratory, Max Planck Institute for
Astrophysics, Max Planck Institute for Extraterrestrial Physics, New
Mexico State University, New York University, Ohio State University,
Pennsylvania State University, University of Portsmouth, Princeton
University, the Spanish Participation Group, University of Tokyo,
University of Utah, Vanderbilt University, University of Virginia,
University of Washington, and Yale University.

B.T.M. is supported by the National Science Foundation Graduate Research
Fellowship under Grant No. DGE‐1144469. 
J.A.J. is supported by generous grants from the David and Lucile Packard
Foundation and the Alfred P. Sloan Foundation.



{\it Facilities:} \facility{Kepler}%, \btmtodo{More...}



\bibliography{exopapers}



\end{document}

 
