% The rules:
% I guess same as DFM because he'll complain if we don't use the 
% character limit.
% New sentences on each line.
% Citation format is \citet{AuthornameYY}.
% References go into papers.bib alphabetically, with ties broken
% chronologically.

%\documentclass[]{aastex}
\documentclass{emulateapj}
%\documentclass[]{emulateapj}

\bibliographystyle{apj}
\usepackage{subfigure}
\usepackage{url}
\usepackage{hyperref}
\usepackage{datetime}
\usepackage{longtable}
\usepackage{natbib}
\usepackage{amsmath}
\usepackage{ulem}
\usepackage{bm}
\usepackage[usenames,dvipsnames]{xcolor}

%\usepackage{pdflscape}

% FIGSET-MACROS-BEGIN
\newcommand{\noprint}[1]{}
\newcommand{\figsetstart}{{\bf Fig. Set} }
\newcommand{\figsetend}{}
\newcommand{\figsetgrpstart}{}
\newcommand{\figsetgrpend}{}
\newcommand{\figsetnum}[1]{{\bf #1.}}
\newcommand{\figsettitle}[1]{ {\bf #1} }
\newcommand{\figsetgrpnum}[1]{\noprint{#1}}
\newcommand{\figsetgrptitle}[1]{\noprint{#1}}
\newcommand{\figsetplot}[1]{\noprint{#1}}
\newcommand{\figsetgrpnote}[1]{\noprint{#1}}
% FIGSET-MACROS-END
\usepackage{color}

\newcommand{\ron}{\color{red}} 
\newcommand{\bon}{\color{blue}} 
\newcommand{\gon}{\color{green}} 
\newcommand{\coff}{\color{black}\,}

\newcommand{\rprs}{{$R_p/R_{\star}$}}

\newcommand{\eg}{{\it e.g.}}
\newcommand{\ie}{{\it i.e.}}
\newcommand{\project}[1]{\textsl{#1}} 
\newcommand{\kep}{\project{Kepler}}
\newcommand{\KT}{\project{K2}}
\newcommand{\vsini}{{$V \sin i$}}
\newcommand{\teff}{$T_{\rm eff}$}
\newcommand{\kms}{{km\,s$^{-1}$}}
\newcommand{\gcc}{{g\,cm$^{-3}$}}
\newcommand{\rstar}{{$R_\star$}}
\newcommand{\rhostar}{{$\rho_\star$}}
\newcommand{\mearth}{{M$_\oplus$}}
\newcommand{\rearth}{{R$_\oplus$}}
\newcommand{\rsun}{{R$_\odot$}}
\newcommand{\msun}{{M$_\odot$}}
\newcommand{\mjup}{{M$_\textrm{Jup}$}}
\newcommand{\rjup}{{R$_\textrm{Jup}$}}


\newcommand{\mstar}{{$M_\star$}}
\newcommand{\logg}{{log(g)}}
\newcommand{\mh}{{[M/H]}~}
\newcommand{\feh}{{[Fe/H]}~}
\newcommand{\afe}{{[$\alpha$/Fe]}~}

\newcommand{\is}{importance sampling}
\newcommand{\Is}{Importance sampling}
%\newcommand{\h2ok2}{{$\rm H_2O-K2$}}

\newcommand{\todo}[3]{{\color{#2} \emph{#1} TO DO: #3}}
\newcommand{\btmtodo}[1]{\todo{BEN}{red}{#1}}



\begin{document}
\title{(Working Title) Properties of the \KT\ Campaign 1 Planets, 
False Positive Probabilities, and Confirmation of $N$ Systems. You 
Were Warned it's a Working Title.}

\author{
% The Heavy Lifters
Benjamin T. Montet\altaffilmark{1,2}, 
Daniel Foreman-Mackey\altaffilmark{3},
Timothy D. Morton,
% The Benefactors
John Asher Johnson,
David W. Hogg,
% The Observers, as individuals
Brendan P. Bowler,
Andrew W. Mann,
David W. Latham,
% The observers, as a group
And Whoever DaveL Tells Us to Add as TRES Team Members
}

\email{btm@astro.caltech.edu}

\altaffiltext{1}{Cahill Center for Astronomy and Astrophysics, California Institute of Technology, 1200 E. California Blvd., MC 249-17, Pasadena, CA 91106, USA}
\altaffiltext{2}{Harvard-Smithsonian Center for Astrophysics, 60 Garden
Street, Cambridge, MA 02138, USA}
\altaffiltext{3}{Department of Astronomy, University of Michigan, Ann Arbor, MI 48109, USA}


\date{\today, \currenttime}

\begin{abstract}
\btmtodo{This is where the abstract will go}
\end{abstract}

\keywords{\btmtodo{Keywords}}

\maketitle

\section{Introduction}


\section{Stellar Properties}
\subsection{Photometry}

With the exception of one star in our sample (EPIC 201912552), we do not have
spectroscopic data with which to characterize the stellar properties. 
Instead, we rely on photometry.
For each system, we query the VizieR database of astronomical catalogues
\citep{Ochseinbein00}. 
We record the $B$, $V$, $g'$, $r'$, and $i'$ magnitudes and their
uncertainties from the AAVSO Photometric all-sky survey (APASS) DR6 
\btmtodo{Cite: Who?}, as reported in the UCAC4 Catalogue \citep{Zacharias12}.
We also record the $J$, $H$, and $K$ magnitudes and their uncertainties 
as found in the 2MASS All-Sky Catalog of Point Sources \citep{Cutri03}
and the four WISE magnitudes and uncertainties from the ALLWise Data
Release \citep{Cutri13}.
For all except two of our targets, the $W4$ band is only an upper limit.
These data are reported in Table \ref{Tab:Photometry}.

\btmtodo{Make this table!}

\subsection{Stellar Models}
To convert the observed photometric data into physical properties for each
star, we employ the use of two stellar models. 
We first download the isochrone grids from the 2012 updated photometric
systems for the relevant bandpasses from the Dartmouth Stellar Evolution
Database \citep{Dotter08}. 
We then interpolate the isochrones onto a grid uniformly spaced in
stellar mass (from 0.20 to 2.0 \msun), metallicity (from \feh$=-1.5$ to 
\feh$=+0.5$), age (from 1.0 to 14.0 Gyr), and log distance (from 10 pc to 
1000 pc). 
In our interpolation, we assume \afe$=0.0$ and $Y=0.2741$, the solar value.
We then compare the expected apparent magnitudes to our observed photometry
to measure the likelihood of each grid point.
For each parameter, we then marginalize our likelihood over all other 
parameters in order to estimate the value of the physical properties
for each star.

Such a scheme enables us to infer the statistical uncertainties on the
mass, radius, and effective temperature.
However, we remain ignorant of any systematic uncertainties that may be
induced by our reliance on the Dartmouth models.
To this end, we repeat this procedure with the Padova database of stellar
evolutionary tracks and isochrones \btmtodo{Cite: Who? Claims to be in
file headers}. 
Comparing two models will not help us understand any systematic errors
of the two are biased in the same way, but in general it will allow us
to calculate a more accurate assessment of our true uncertainties than 
relying on one model alone.

\btmtodo{Padova parameters, information}

\btmtodo{Show plot}

The Padova models consistently produce radii which are \btmtodo{VALUE\%} 
larger than those predicted by the Dartmouth models. 
There is some evidence that the Dartmouth models may underpredict 
radii by $\sim 15\%$ when compared to other methods \citep{Newton15,
Montet15}. 
Given that these stars form the majority of our sample, this discrepancy
is perhaps not surprising.

\btmtodo{Then we combine them into one set of stellar properties?}

\subsection{SpeX Spectroscopy}


\section{False Positive Analysis}
\subsection{Adaptive Optics Imaging}

\subsection{Archival Imaging}

\subsection{Known Background Stars}

\subsection{False Positive Rates (and confirmations?)}
This is Tim.

\section{Potentially Interesting Systems}
\subsection{The One Around the M dwarf}

\subsection{Some Multi (possible multi of these}

\subsection{Others? Really bright ones amenable to followup?}

\section{Results and Discussion}

Something about Multis and TTVs. 




\acknowledgements
We thank [People] for [things] which improved the quality of this manuscript.


We are grateful to the entire Kepler team, past and present. 
Their tireless efforts were all essential to the tremendous success of the mission and the successes of K2, present and future.


Some of the data presented in this paper were obtained from the Mikulski
Archive for Space Telescopes (MAST). 
STScI is operated by the Association of Universities for Research 
in Astronomy, Inc., under NASA contract NAS5--26555. 
for MAST for non--HST data is provided by the NASA Office of Space 
Science via grant NNX13AC07G and by other grants and contracts. 
This paper includes data collected by the \kep\ mission. 
Funding for the \kep\ mission is provided by the NASA Science 
Mission directorate.

B.T.M. is supported by the National Science Foundation Graduate Research
Fellowship under Grant No. DGE‐1144469. 
J.A.J. is supported by generous grants from the David and Lucile Packard
Foundation and the Alfred P. Sloan Foundation.



{\it Facilities:} \facility{Kepler}%, \btmtodo{More...}



\bibliography{exopapers}



\end{document}

 
