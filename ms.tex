% The rules:
% I guess same as DFM because he'll complain if we don't use the 
% character limit.
% New sentences on each line.
% Citation format is \citet{AuthornameYY}.
% References go into papers.bib alphabetically, with ties broken
% chronologically.

%\documentclass[]{aastex}
\documentclass{emulateapj}
%\documentclass[]{emulateapj}

\bibliographystyle{apj}
\usepackage{subfigure}
\usepackage{url}
\usepackage{hyperref}
\usepackage{datetime}
\usepackage{longtable}
%\usepackage{lscape}
\usepackage{natbib}
\usepackage{amsmath}
\usepackage{ulem}
\usepackage{bm}
\usepackage[usenames,dvipsnames]{xcolor}

%\usepackage{pdflscape}

% FIGSET-MACROS-BEGIN
\newcommand{\noprint}[1]{}
\newcommand{\figsetstart}{{\bf Fig. Set} }
\newcommand{\figsetend}{}
\newcommand{\figsetgrpstart}{}
\newcommand{\figsetgrpend}{}
\newcommand{\figsetnum}[1]{{\bf #1.}}
\newcommand{\figsettitle}[1]{ {\bf #1} }
\newcommand{\figsetgrpnum}[1]{\noprint{#1}}
\newcommand{\figsetgrptitle}[1]{\noprint{#1}}
\newcommand{\figsetplot}[1]{\noprint{#1}}
\newcommand{\figsetgrpnote}[1]{\noprint{#1}}
% FIGSET-MACROS-END
\usepackage{color}

\newcommand{\ron}{\color{red}} 
\newcommand{\bon}{\color{blue}} 
\newcommand{\gon}{\color{green}} 
\newcommand{\coff}{\color{black}\,}

\newcommand{\rprs}{{$R_p/R_{\star}$}}
\newcommand{\ars}{{$a/R_{\star}$}}

\newcommand{\eg}{{\it e.g.}}
\newcommand{\ie}{{\it i.e.}}
\newcommand{\project}[1]{\textsl{#1}} 
\newcommand{\kep}{\project{Kepler}}
\newcommand{\KT}{\project{K2}}
\newcommand{\Ci}{Campaign~1}
\newcommand{\vsini}{{$V \sin i$}}
\newcommand\teff{\ensuremath{T_\text{eff}}}
\newcommand{\kms}{{km\,s$^{-1}$}}
\newcommand{\gcc}{{g\,cm$^{-3}$}}
\newcommand{\rstar}{{$R_\star$}}
\newcommand{\rhostar}{{$\rho_\star$}}
\newcommand{\mearth}{{M$_\oplus$}}
\newcommand{\rearth}{{R$_\oplus$}}
\newcommand{\rsun}{{R$_\odot$}}
\newcommand{\msun}{{M$_\odot$}}
\newcommand{\mjup}{{M$_\textrm{Jup}$}}
\newcommand{\rjup}{{R$_\textrm{Jup}$}}


\newcommand{\mstar}{{$M_\star$}}
\newcommand{\logg}{{log(g)}}
\newcommand{\mh}{{[M/H]}~}
\newcommand{\feh}{{[Fe/H]}~}
\newcommand{\afe}{{[$\alpha$/Fe]}~}

\newcommand{\paperip}{\citep{Foreman-Mackey15}}
\newcommand{\paperit}{\citet{Foreman-Mackey15}}

\newcommand{\is}{importance sampling}
\newcommand{\Is}{Importance sampling}
%\newcommand{\h2ok2}{{$\rm H_2O-K2$}}

\newcommand{\todo}[3]{{\color{#2} \emph{#1} TO DO: #3}}
\newcommand{\btmtodo}[1]{\todo{BEN}{red}{#1}}
\newcommand{\anytodo}[1]{\todo{ANYONE}{NavyBlue}{#1}}

\newcommand{\Nfp}{$X$}
\newcommand{\Nvalidated}{$Y$}
\newcommand{\Nlimbo}{$Z$}

\begin{document}



\title{(Working Title) Stellar and Planetary Properties of \KT\ \Ci\
Candidates and Confirmation of $N$ Systems, (Including a Planet Receiving 
Earth-like Insolation?)}

\author{
% The Heavy Lifters
Benjamin T. Montet\altaffilmark{1,2}, 
Daniel Foreman-Mackey\altaffilmark{3},
Timothy D. Morton,
% The Benefactors
John Asher Johnson,
David W. Hogg,
% The Observers
% (ranked by integration time x collecting area)
Brendan P. Bowler,
David W. Latham,
Allyson Bieryla,
Andrew W. Mann,
And Any TRES Team Members We Should Add
}

\email{btm@astro.caltech.edu}

\altaffiltext{1}{Cahill Center for Astronomy and Astrophysics, California Institute of Technology, 1200 E. California Blvd., MC 249-17, Pasadena, CA 91106, USA}
\altaffiltext{2}{Harvard-Smithsonian Center for Astrophysics, 60 Garden
Street, Cambridge, MA 02138, USA}
\altaffiltext{3}{Department of Astronomy, University of Michigan, Ann Arbor, MI 48109, USA}


\date{\today, \currenttime}

\begin{abstract}
Recently, \paperit{} presented a list of 36 planet candidates orbiting
31 stars in \KT\ \Ci. 
In this paper, we present stellar and planetary properties for all systems.
We combine archival imaging, ground-based seeing limited survey data, and
adaptive optics imaging with a detailed transit analysis scheme to confirm 
\btmtodo{Nconfirm} as planets and \btmtodo{NFP} as false positives
at the \btmtodo{confidence} level.
Of particular interest is EPIC 201912552, a bright (K=8.9) M2 dwarf
hosting a \anytodo{Radius} planet with $T_{eq} = 275 \pm 15$ K and an 
orbital period of 33 days.
\anytodo{Anything we should add?}
\end{abstract}

\keywords{\btmtodo{Keywords}}

\maketitle

\section{Introduction}
The \kep\ telescope \citep{Borucki10} has led to a revolution in stellar and planetary 
astrophysics, with 7305 ``objects of interest'' and 4173 ``planet candidates''
discovered to date \citep{Borucki11a, Borucki11b, Batalha13, Burke14, Rowe15, Mullaly15}.
The fidelity of this sample is high, so most of these planet candidates are truly planets
\citep{Morton11b, Fressin13}.
The demise of two reaction wheels on the spacecraft led to a repurposing 
of the spacecraft into the \KT\ mission, in which the telescope points at
fields near the ecliptic plane for $\sim 75$ days at a time \citep{Howell14}.
In this observing strategy, two axes of motion of the spacecraft are 
controlled by the two remaining reaction wheels, while the roll of the 
spacecraft is balanced with solar radiation pressure and quasiperiodic 
thruster firing.
The result of this observing strategy is that the targets appear to drift
at the rate of $\sim 1\arcsec$ s$^{-1}$, with rapid corrections that occur
approximately once every six hours.
Over the full duration of each campaign, the targets appear to remain at 
approximately the same location on the detector.
Both the slow drift and the corrections are observable by eye, and by viewing the
motions of each star \citep{Barentsen15} it is apparent that \KT\ is working as 
advertised.
Therefore, apertures can be defined around each star and aperture photometry can be
defined in the usual way.

\KT\ light curves produced with aperture photometry contain substantial pointing-induced
photometric variations caused by the star's apparent motion over a poorly-defined flat 
field. 
Worse yet, these variations occur on timescales similar to transit signals, potentially
masking the observational signature of a planet passing between \kep\ and the host star.

There has been considerable effort to recover these signals.
\citet{Vanderburg14} extract aperture photometry and decorrelate with image centroid
position, ``correcting'' the data to ``remove'' the effects of the spacecraft motion.
These data were used to confirm the first planet discovered in \KT\ \citep{Vanderburg15}.
A similar technique was employed by \citet{Armstrong15a} to identify potential eclipsing
binary systems and to identify a transiting system very near a 3:2 period commensurability
\citep{Armstrong15b}.
\citet{Aigrain15} and \citet{Crossfield15} use a Gaussian Process model for the measured
flux, with pointing measurements as the inputs, and then “detrend” the data with that model.
The latter of these detected the first multi-planet system in \KT\ data.

What is common to all of these methods are that removal of systematics is considered a 
step to be undertaken before the search for planets. 
Under this strategy, it is implicitly assumed that the systematics are removed perfectly,
while retaining all of the astrophysical signal. 
Of course, it is impossible to perfectly separate the astrophysical and instrumental 
signal, and such a technique is prone to either over-fitting, in which some of the 
astrophysical signal is also removed, or under-fitting, in which some instrumental noise
remains.
A better strategy is to simultaneously fit both the signal and the noise, as is commonly
done in cosmology and, increasingly, in radial velocity searches for planetary systems
\citep[e.g.][]{Ferreira00, Boisse11, Haywood14, Grunblatt15}. 

\paperit{} simultaneously fit both the systematics and potential planetary transit signals
in a search for transiting planets. 
The authors of that paper assume that the dominant trends in the observed stellar light
curves are caused by spacecraft motion and are shared among many stars. \anytodo{Among,
right? Not between?}
They then run PCA on all stars to measure the dominant modes, modeling each star as a 
linear combination of 150 of these ``eigen light curves'' and a transit signal.
This method enables fitting without over-fitting, and also permits marginalization over
uncertainties induced by the noise model.
Therefore, any uncertainties in the systematics can be propogated into uncertainties in
detected planet parameters, instead of assuming the systematics are understood perfectly.
Using this technique, \paperit{} detect 36 planet candidates orbiting 31 stars in \KT\
\Ci\ data.

In \paperit, only transit properties are provided, not absolute parameters about the
planet or the star. 
Additionally, the authors follow the convention of the \kep\ team to include any
transit event as a candidate system rather than a false positive if a secondary eclipse
is not detected: there is no enforced upper limit on the allowed planet radius.
The authors intentionally make no effort to separate astrophysical false positives from
true transiting planets.

In this paper, we present stellar and planetary parameters for each system.
Given these parameters and the transit data, as well as archival imaging, ground-based
seeing limited survey data, and adaptive optics imaging, we are able to confirm 
\anytodo{Number} of these systems as transiting planets at the \anytodo{Sigma} confidence
level. We also are able to confirm \anytodo{Number} of these systems as false positives.

This paper is organized as follows. 
In \textsection2, we develop stellar properties through
photometric and spectroscopic data.
In \textsection3, we combine the derived stellar properties with \KT\ data to infer planet
candidate properties.
In \textsection4, we combine adaptive optics and radial velocity observations with both
archival and modern ground-based, seeing limited survey data and an analysis of the
transit parameters to calculate false positive probabilities.
In \textsection5, we discuss potentially interesting systems.
In \textsection6, we summarize and discuss our results.


\section{Stellar Properties}
\subsection{Photometry}

With the exception of one star in our sample (EPIC 201912552), we do not have
spectroscopic data with which to characterize the stellar properties. 
Instead, we rely on photometry.
For each system, we query the VizieR database of astronomical catalogues
\citep{Ochseinbein00}. 
We record the $B$, $V$, $g'$, $r'$, and $i'$ magnitudes and their
uncertainties from the AAVSO Photometric all-sky survey (APASS) DR6 
\btmtodo{Cite: Who?}, as reported in the UCAC4 Catalogue \citep{Zacharias12}.
We also record the $J$, $H$, and $K$ magnitudes and their uncertainties 
as found in the 2MASS All-Sky Catalog of Point Sources \citep{Cutri03}
and the $W1-W3$ WISE magnitudes and uncertainties from the ALLWise Data
Release \citep{Cutri13}.
For all except two of our targets, the $W4$ band is only an upper limit,
and in the remaining two cases, the photometric uncertainity in $W4$ is at least an 
order of magnitude larger than those in $W1-W3$, so we neglect $W4$.
These data are reported in Table 1.


\subsection{Stellar Models}
To convert the observed photometric data into physical properties for each
star, we employ the use of stellar models. 
We first download the isochrone grids from the 2012 updated photometric
systems for the relevant bandpasses from the Dartmouth Stellar Evolution
Database \citep{Dotter08}. 
We then interpolate the isochrones onto a grid uniformly spaced in
stellar mass (from 0.20 to 2.0 \msun), metallicity (from \feh$=-1.5$ to 
\feh$=+0.5$), age (from 1.0 to 14.0 Gyr), and log distance (from 10 pc to 
1000 pc). 
In our interpolation, we assume \afe$=0.0$ and $Y=0.2741$, the solar value.
We then compare the expected apparent magnitudes to our observed photometry
to measure the likelihood of each grid point.
For each parameter, we then marginalize our likelihood over all other 
parameters in order to estimate the value of the physical properties
for each star.

Such a scheme enables us to infer the statistical uncertainties on the
mass, radius, and effective temperature.
However, we are subject to biases induced by systematics in the models themselves.
There is some evidence that the Dartmouth models may underpredict 
radii by $\sim 15\%$ when compared to other methods \citep{Newton15,
Montet15}. 
Such an effect may be the result of the Dartmouth model reliance on BT-Settl
atmospheres, which have been shown to predict near-IR colors that are too blue
\citep{Thompson14}.
We find that by measuring stellar parameters without any WISE data (that is, using
only $BVgriJHK$), we largely reproduce the same effective temperatures as when
the WISE data is included. 
However, at the low-mass end including the WISE data systematically increases the 
measured radius and increases \logg, as shown in \btmtodo{Figure}.
Suggestively, spectroscopic stellar parameters of EPIC 201912552 
(\textsection\ref{Spexobs}) suggest the derived photometric properties better 
represent the true physical parameters when the WISE data is included. 


\btmtodo{Show plot}



\subsection{SpeX Spectroscopy}
\label{Spexobs}


A near-infrared spectrum of EPIC 201912552 was obtained using the updated SpeX 
(uSpeX) spectrograph \citep{Rayner03} on the NASA Infrared Telescope Facility 
(IRTF) on January 29 2015 (UT). 
SpeX observations were taking using the short cross-dispersed mode and the
0.3$\times15\arcsec$ slit, which provides simultaneous coverage from 0.7 
to 2.5$\mu$m at $R\simeq2000$. 
The target was observed at two positions along the slit to subsequently subtract 
the sky background. Eight spectra were taking following this pattern, which provided 
a final S/N of $>150$ per resolving element. 
The spectrum was flat fielded, extracted, wavelength calibrated, and stacked 
using the \textit{SpeXTool} package \citep{Cushing04}. 
An A0V-type star was observed immediately after 
the target, which was used to create a telluric correction using the 
\textit{xtellcor} package \citep{Vacca03}.

An optical spectrum was obtained using the SuperNova Integral Field Spectrograph
\citep[SNIFS,][]{Aldering02,Lantz04} on the University of Hawai'i 
2.2m telescope on the night of January 30 2015. 
SNIFS provides simultaneous coverage from 3200\AA--9700\,AA\ at a resolution 
of $\simeq1000$. Final S/N of the spectrum was $>100$ pre resolving element 
in the red ($\sim6000$\AA). 
Details of the SNIFS reduction, including dark, bias, and flat-field corrections,
cleaning the data of bad pixels and cosmic rays, and extraction of the 
one-dimensional spectrum are described in \citet{Bacon01} and 
\citet{Aldering06}. 
Flux calibration was performed using a separate pipeline described in \citet{Mann15}. 

\teff\ was calculated by comparing our optical spectra with the CFIST
suite\footnote{http://phoenix.ens-lyon.fr/Grids/BT-Settl/CIFIST2011/} of the BT-SETTL
version of the PHOENIX atmosphere models \citep{Allard13}, which gave a temperature
of 3503 $\pm$ 60\,K. 
More details of this procedure are given in \citet{Mann14} and
\citet{Gaidos14}. 
This method was used because it is known to accurately reproduce empirical 
\teff\ values from long-baseline optical interferometry \citet{Boyajian12}. 

Metallcity was determined using the procedures from \citet{Mann13a}, in which the
authors provide empirical relations between atomic features and M dwarf
metallicity, calibrated using wide binaries. 
We adopted the weighted mean of the $H-$ and $K-$band calibrations, 
which yielded a metallicity of 0.09$\pm$0.09.

We combined the derived \teff\ and [Fe/H] values with the empirical 
\teff-[Fe/H]-$R_*$ relation from \citet{Mann15} to compute a radius. 
Accounting for measurement and calibration errors in [Fe/H] and \teff\ we calculated 
a radius 0.394$\pm0.038R_\odot$. 
We use these parameters instead of the derived photometric properties for this target.

The full list of stellar parameters adopted in this paper is included in
Table 2.
% For some reason I can't get it to identify the table \ref here. 
% If anyone can figure it out, be my guest!

\section{Planet Properties}

This is Dan.

(Include comment here about how the planet parameters assume no dilution. 
Except for the two that we have AO detected binaries for, probably?
Brendan should have us flux ratios by the end of the day.
In those cases, I suppose we need to include planet properties in both cases
in which the transit is around either the more or less massive star.)

\section{False Positive Analysis}
As has been well known from the early days of transit searches, it is 
possible for an eclipsing binary star to masquerade as a transiting 
planet if (a) it is a highly grazing eclipse, or (b) the binary system
comprises only a small fraction of the total light in the photometric
aperture, resulting in a diluted eclipse depth.
When possible, such astrophsycial false positive scenarios are traditionally
ruled out by detailed follow-up observations, such as high-resolution
imaging or radial-velocity measurements.  
However, the \kep\ mission, with its thousands of planet candidates
around mostly faint stars, necessitated a paradigm shift---a move 
toward probabilistic interpretation of transit signals, rather than 
comprehensive follow-up of each individual candidate.

\citet{Morton12} described an automated method to calculate the
probability that a planet candidate might be caused by an
astrophysical false positive.  This method uses population simulations
of false positive scenarios and true transiting planets to quantify
the typical shape of each, and combines prior information about the
populations of field stars and multiple star systems in order to
determine the probability that the observed signal may be a false
positive.  Similar in spirit to other published methods of
probabilistic validation, such as Blender \citep{} and PASTIS
\citep{}, it has the advantage of being computationally less demanding
and fully automated, and thus easily applied in batch to a large
number of candidates.
 
In this work, we use
\texttt{vespa}\footnote{\url{http://github.com/timothydmorton/vespa}},
a new implementation of the \citet{Morton12} procedure, to calculate
the false positive probabilities of each of these K2 candidates,
supplemented in a few cases by high-resolution imaging, where
available.  Table \ref{table:FPP} summarizes the results of these
calculations, presenting the relative probability for each candidate
to be caused by any of three false positive scenarios: an undiluted
eclipsing binary (EB), a hierarchical triple eclipsing binary (HEB),
and a chance-aligned background(/foreground) eclipsing binary (BEB).
We find that \Nfp\ of the presented candidates have false positive
probabilities (FPPs) $>$80\%; these we consider to be likely false
positives.  On the other hand, we find that \Nvalidated\ candidates
have FPP $<$ 1\%; these we consider to be validated planets.  The
remaining \Nlimbo\ candidates have FPPs neither high enough to discard
nor low enough to securely validate.

\subsection{Adaptive Optics Imaging}

This is Brendan.




\subsection{Known Background Stars}
The P3K AO system has a field of view of $3.84 \times 3.84$ arcseconds.
\btmtodo{Check with Brendan to make sure he used this mode}.
Each \KT\ pixel is a square of $3.98 \times 3.98$ arcseconds. 
A background eclipsing binary within a few \KT\ pixels of our target 
stars could mimic a transit signal while evading detection by P3K.
Such wide eclipsing binaries should appear in seeing-limited ground-based
surveys.

To investigate the possibility that such wide companions exist,
we query the ninth data release of the Sloan Digital Sky Survey 
\citep[SDSS DR9,][]{Ahn12}. 
For each target, we return
all stars within 12\arcsec with $\Delta r < 8$ relative to each object
of interest.
Of the 31 stars in our sample, four have such a companion.
Unlike the original \kep\ field, the field for \KT\ \Ci\ is 
well out of the galactic plane, so the rate of giant, distanct background
stars is significantly lower.
We include a list of these potential contaminants as Table 
\ref{Table:BGstars}.

%Table:BGstars
% Primary    RA Dec Separation DeltaR         MaxDepth
% 201295312  &  &   &          & 7.10 \pm 0.10 & 0.75
% 201465501  &  &   &          & 7.90 \pm 0.25 & 0.36
% 201546283  &  &   &          & 5.87 \pm 0.06 & 2.31
% 201779067  &  &   &          & 4.21 \pm 0.03 & 10.56

In Table \ref{Table:BGstars}, the ``maximum depth'' column represents
the maximum observed ``transit'' depth if the transit were actually caused
by a total eclipse of the hypothetical background binary system, inducing
a 50\% flux decrement in the background star's apparent brightness.

For EPIC 201465501 and 201779067, the observed transit depth is 
larger than each of these values, so we reject the possibility that
these events could be the eclipses of the background binaries detected
here. 

For EPIC 201546283, the observed transit depth is nearly identical to 
the maximum false positive depth. 
The photometry for this potential contaminant is consistent with that
of an M3V star; for this object to cause the observed eclipse it must
be a total eclipse of two M3V stars. 
However, in that case, the observed event would be exactly V-shaped,
with second and third contact occurring simultaneously. 
Since the transit is not V-shaped, it must be the result of
a relatively small body passing in front of a relatively large body. 
Therefore, it cannot be caused by this observed background dwarf star.

For EPIC 201295312, the transit depth is significantly shallower than
the maximum false positive depth, so we cannot immediately
reject the hypothesis that the transit is actually a false positive 
induced by a background eclipsing binary. \anytodo{Can we place limits
on this possibility from the lack of a centroid shift?}

SDSS is 95\% complete at $r=22.2$ and the telescope has a point
spread function of $1\farcs4$. In all cases except the four above, we 
treat the SDSS data as providing a contrast curve at wide separations
down to a limiting magnitude of $r=22.2$.

\subsection{Archival Imaging}

For the stars with AO nondetections, there is still the
possibility that a background binary could be positioned
directly behind the target star, evading detection.
The probability is small, given the $0\farcs1$ diffraction limit
of P200 at $2\mu$m, but nonzero.
We can rule out the possibility of such chance alignments with 
archival imaging data. 

\btmtodo{Method}

\btmtodo{Plots showing POSS observations.}

For these targets, we can extend our contrast curves to zero 
present-day orbital separation and rule out the possibility that
these transit events are caused by background eclipsing binaries.
By combining present-day seeing-limited photometric survey data, 
adaptive optics imaging, and archival photometry, the only stellar companions 
we would not detect would be those that are gravitationally bound to the target 
star and positioned in their orbits so that their projected separation is
smaller than the diffraction limit of the P200 telescope.
Such an alignment would require the orbital inclination of the binary to be 
nearly $90^\circ$ and the phase $\varpi + \theta \approx \pi/2$ or $3\pi/2$.
While we cannot fully rule out this possibility, such an alignment is unlikely
to occur, and highly unlikely to cause enough dilution to call the planetary
nature of these signals into question.




\subsection{TRES Radial Velocities}

We observed EPIC 201912552 on (all UT) February 04 2015 and February 24 2015 
with the Tillinghast Reflector Echelle
Spectrograph (TRES) on the 1.5 m Tillinghast Reflector at the Fred L.
Whipple Observatory. 
These dates were chosen to be near the times of largest RV variations, 
corresponding to phases of 0.72 and 0.32 relative to the time
of transit.
The spectra were taken with a resolving power of $R=44,000$ and 
integration times ranging from 2800 to 3600 seconds, resulting in 
signal-to-noise ratios between 17 and 29 per resolution element.

The spectra were extracted as described in \citet{Buchhave10}. 
The relative RVs were derived by cross-correlating the spectra against the
strongest observed spectrum (in this case, the first) over the wavelength
range 4700 - 6800 Angstroms. 
We selected 19 echelle orders in the analysis, being careful to reject
orders with telluric absorption lines, fringing in the far red and those
with very low SNR in the blue.

The two observed spectra have RVs that differ by $47.1 \pm 41.9$ m s$^{-1}$.
A deviation at this magnitude can only be explained by the presence of 
a planetary system and instrumental noise, eliminating the possibility the
transiting planet is actually a false positive.



\subsection{False Positive Rates (and confirmations?)}
This is Tim.

\section{Potentially Interesting Systems}
\subsection{A Mini-Neptune with Earthlike Insolation}

Of particular interest is EPIC 201912552. 

By combining archival and modern seeing-limited data with adaptive optics 
imaging, we can exclude the possibility these transit events are caused by 
a background eclipsing binary.
The apparent transits must be caused by an object co-moving with EPIC 201912552;
radial velocities eliminate the possibility the companion is nonplanetary.
Therefore, we confirm the planetary nature of this system.

This star is an M2.8 dwarf at a distance of $34\pm4$ pc.
Of our planet candidate hosts, only EPIC 201367065 \citep[originally
discovered by][]{Crossfield15} is brighter in $K$-band.
This star is only 0.1 magnitudes fainter in $K$ than GJ\,1214 
\citep{Charbonneau09}.
Due to the relative brightness of the host star, this target is likely
to become a prime target for atmospheric characterization studies
and is ideal as a target for future space-based missions such as JWST.

Unlike GJ\,1214b, EPIC 201912552b is not highly irradiated. 
Instead, it is at an reduced semimajor axis \ars$ = 82.1 \pm 8.6$.
Its equilibrium temperature is then, assuming zero albedo, $T_{eq} = 275 \pm 15$
K, meaning its bulk insolation is $135 \pm 30$ percent that of the Earth's.
Although the planet is likely too large to be rocky \citep{Rogers14}, 
its atmosphere is likely to be the focus of many future observations, providing
a cool analogue to the highly irradiated planets of a similar size found by 
\kep.



\subsection{EPIC 201465501?}
Is this a subsection? It's a star with a mass of ~0.2 Msun and a candidate with
a period of 18 days. I suppose it depends on the FPP.

\subsection{An Instrumental False Positive}

EPIC 201555883 was originally listed in \paperit\ as a planet candidate.
After submission of that work, this object of interest was identified as 
having a transit ephemeris consistent with that of EPIC 201569483, 
which we identify as a likely eclipsing binary due to the observed transit
depth \anytodo{Right? Right?}. 
We conclude the observed transit signal from EPIC 201555883 is likely the result 
of an instrumental artifact.

Such effects are not uncommon in \kep\ data. 
\citet{Batalha13} found $\sim 25$ planet candidates to be false positives based on matching
ephemerides between candidates. 
In that paper, the authors limited their search to candidates within $20\arcsec$ of other
objects of interest. 
\citet{Coughlin14} expanded on this method, identifying 685 KOIs as false positives and 
outlining four physical reasons why these anomalies may occur.
While EPIC 201555883 is likely a false positive, it does not appear that it can neatly be
explained by any one of the effects described by \citet{Coughlin14} alone. 
This artifact should be carefully studied in the future, as it may help illuminate other 
potential detector artifacts, both in \kep\ and \KT\ data.


\subsection{Multiple Planet Systems}
Five of the systems identified by \paperit\ are candidates to be multiply-transiting
systems. 
One of these is EPIC 201367065, a three-planet system originally announced by
\citet{Crossfield15}. 
Another of these is EPIC 201505350 \citep{Armstrong15b}, a two-planet system with the
orbital periods of the two planets near a 3:2 period commensurability.
The remaining three are all representative of the multiple-planet systems observed by
\kep \citep{Lissauer11b, Fabrycky14}.
Two of the systems are near a period commensurability and all three consist of 
mini-Neptune sized planets.
Like in the \kep\ field, the false positive rate we measure for the multiple planet sample
here ($0/10$) is lower than for the single systems \citep{Lissauer12, Rowe14}.


We do not detect any significant transit timing variations (TTVs) in any of these systems from 
the \KT\ data alone. 
EPIC 201338508 would be expected to have a TTV period of 117 days, but 
is likely too far from commensurability to have an observable TTV signal. 
EPIC 201445392 is expected to have a TTV period of 234 days, so this system may be a candidate
for additional follow-up to constrain the system masses dynamically. 
The transiting planets orbiting EPIC 201754305 are near a 5:2 period commensurability. 
There is no evidence from \kep\ of an abundance of planets near this period ratio, and so this 
may be coincidence.
Follow-up observations may be warranted to search for an additional planet in this system 
forming a resonant chain, similar to those observed around other stars \citep[e.g.][]{Swift13,
Campante15}.

\subsection{Systems Orbiting Bright Stars}

One of the primary goals of \KT\ is the detection of transiting planets around bright stars that
can be followed up from the ground or with future space-based observatories such as JWST 
\citep{Howell14}. 
Of our sample, two systems orbit stars with $K < 9$: EPIC 201367065 \citep{Crossfield15} and
EPIC 201912552. 
An additional four targets orbit stars with $K <10$. 
These targets are ideal for ground-based followup, and may be useful targets for Spitzer and 
JWST to probe planetary atmospheres.

\section{Results and Discussion}

We have presented stellar parameters for all planet candidates systems observed by 
\paperit, confirming \btmtodo{Nconfirm} of the 36 candidates as bona fide planets and 
\btmtodo{NFP} as bona fide false positives. 

With the exception of one object, all of the stellar parameters are derived from comparing 
photometric observations to the Dartmouth stellar evolution models. 
As a result, both the stellar and planet parameters are subject to systematic biases induced
by discrepancies between the models and reality. 


\anytodo{We should have a figure of many subplots, of phase-folded
transits for all the confirmed planets. 
To ease Dan's mind, we will say that they are the product of 
some fidicual transit model and are only presented here for illustrative
purposes.}



\acknowledgements
We thank Eric Agol (UW), Dan Huber (Sydney) [and Other People] for conversations and
suggestions which improved the quality of this manuscript.


We are grateful to the entire Kepler team, past and present. 
Their tireless efforts were all essential to the tremendous success of the mission and the
successes of K2, present and future.


Some of the data presented in this paper were obtained from the Mikulski
Archive for Space Telescopes (MAST). 
STScI is operated by the Association of Universities for Research 
in Astronomy, Inc., under NASA contract NAS5--26555. 
for MAST for non--HST data is provided by the NASA Office of Space 
Science via grant NNX13AC07G and by other grants and contracts. 

This paper includes data collected by the \kep\ mission. 
Funding for the \kep\ mission is provided by the NASA Science 
Mission directorate.

This paper includes data collected by the Sloan Digital Sky Survey.
Funding for SDSS-III has been provided by the Alfred P. Sloan Foundation,
the Participating Institutions, the National Science Foundation, and the
U.S. Department of Energy Office of Science. The SDSS-III web site is
http://www.sdss3.org/.
SDSS-III is managed by the Astrophysical Research Consortium for the
Participating Institutions of the SDSS-III Collaboration including the
University of Arizona, the Brazilian Participation Group, Brookhaven
National Laboratory, Carnegie Mellon University, University of Florida,
the French Participation Group, the German Participation Group, Harvard
University, the Instituto de Astrofisica de Canarias, the Michigan
State/Notre Dame/JINA Participation Group, Johns Hopkins University,
Lawrence Berkeley National Laboratory, Max Planck Institute for
Astrophysics, Max Planck Institute for Extraterrestrial Physics, New
Mexico State University, New York University, Ohio State University,
Pennsylvania State University, University of Portsmouth, Princeton
University, the Spanish Participation Group, University of Tokyo,
University of Utah, Vanderbilt University, University of Virginia,
University of Washington, and Yale University.

B.T.M. is supported by the National Science Foundation Graduate Research
Fellowship under Grant No. DGE‐1144469. 
J.A.J. is supported by generous grants from the David and Lucile Packard
Foundation and the Alfred P. Sloan Foundation.


D.F.M. and D.W.H. were partially supported by the National Science Foundation 
(grant IIS-1124794), the National Aeronautics and Space Administration (grant 
NNX12AI50G), and the Moore–Sloan Data Science Environment at NYU.


{\it Facilities:} \facility{Kepler}%, \btmtodo{More...}



\bibliography{exopapers}

\clearpage
%\LongTables
\begin{deluxetable*}{cccccccccccc}
\tablewidth{0pt}
\tabletypesize{\scriptsize}
\tablecaption{\label{Tab:Photometry} Photometry for all Objects of Interest}
\tablehead{
\colhead{EPIC} &
\colhead{$B_\textrm{APASS}$} &
\colhead{$V_\textrm{APASS}$} &
\colhead{$g_\textrm{APASS}$} &
\colhead{$r_\textrm{APASS}$} &
\colhead{$i_\textrm{APASS}$} &
\colhead{$J_\textrm{2MASS}$} &
\colhead{$H_\textrm{2MASS}$} &
\colhead{$K_\textrm{APASS}$} &
\colhead{W1} &
\colhead{W2} &
\colhead{W3} 
}
\startdata
201208431 & $16.23 \pm 0.05$ & $14.91 \pm 0.03$ & $15.56 \pm 0.04$ & $14.29 \pm 0.07$ & $13.89 \pm 0.12$ & $12.37 \pm 0.02$ & $11.75 \pm 0.02$ & $11.57 \pm 0.02$ & $11.51 \pm 0.02$ & $11.55 \pm 0.02$ & $11.58 \pm 0.20$   \\ 
201257461 & $12.82 \pm 0.03$ & $11.77 \pm 0.01$ & $12.24 \pm 0.04$ & $11.49 \pm 0.01$ & $11.19 \pm 0.02$ & $9.99 \pm 0.02$ & $9.48 \pm 0.02$ & $9.37 \pm 0.02$ & $9.28 \pm 0.02$ & $9.37 \pm 0.02$ & $9.30 \pm 0.04$ \\ 
201295312 & $12.78 \pm 0.04$ & $12.19 \pm 0.12$ & $12.41 \pm 0.03$ & $12.08 \pm 0.09$ & $12.01 \pm 0.21$ & $11.02 \pm 0.03$ & $10.70 \pm 0.02$ & $10.69 \pm 0.02$ & $10.63 \pm 0.02$ & $10.69 \pm 0.02$ & $10.75 \pm 0.12$ \\ 
201338508 & $16.30 \pm 0.07$ & $14.91 \pm 0.03$ & $15.62 \pm 0.05$ & $14.33 \pm 0.02$ & $13.79 \pm 0.05$ & $12.45 \pm 0.03$ & $11.76 \pm 0.02$ & $11.60 \pm 0.02$ & $11.49 \pm 0.03$ & $11.49 \pm 0.02$ & $11.16 \pm 0.13$ \\ 
201367065 & $13.52 \pm 0.06$ & $12.17 \pm 0.01$ & $12.87 \pm 0.03$ & $11.58 \pm 0.02$ & $10.98 \pm 0.17$ & $9.42 \pm 0.03$ & $8.80 \pm 0.04$ & $8.56 \pm 0.02$ & $8.44 \pm 0.02$ & $8.42 \pm 0.02$ & $8.32 \pm 0.02$ \\ 
201384232 & $13.30 \pm 0.05$ & $12.65 \pm 0.04$ & $12.91 \pm 0.05$ & $12.48 \pm 0.06$ & $12.34 \pm 0.07$ & $11.44 \pm 0.02$ & $11.09 \pm 0.02$ & $11.07 \pm 0.02$ & $11.00 \pm 0.02$ & $11.05 \pm 0.02$ & $11.21 \pm 0.16$ \\ 
201393098 & $13.90 \pm 0.04$ & $13.21 \pm 0.03$ & $13.54 \pm 0.06$ & $13.02 \pm 0.04$ & $12.85 \pm 0.05$ & $11.95 \pm 0.02$ & $11.63 \pm 0.02$ & $11.56 \pm 0.02$ & $11.52 \pm 0.02$ & $11.57 \pm 0.02$ & $11.61 \pm 0.21$ \\ 
201403446 & $12.48 \pm 0.02$ & $12.03 \pm 0.02$ & $12.18 \pm 0.01$ & $11.94 \pm 0.05$ & $11.86 \pm 0.04$ & $11.05 \pm 0.03$ & $10.76 \pm 0.02$ & $10.78 \pm 0.02$ & $10.67 \pm 0.03$ & $10.71 \pm 0.02$ & $10.36 \pm 0.07$ \\ 
201445392 & $15.73 \pm 0.02$ & $14.61 \pm 0.03$ & $15.19 \pm 0.04$ & $14.29 \pm 0.02$ & $14.03 \pm 0.07$ & $12.83 \pm 0.03$ & $12.32 \pm 0.03$ & $12.24 \pm 0.03$ & $12.16 \pm 0.02$ & $12.21 \pm 0.02$ & --- \\ 
201465501 & --- & --- & $16.73 \pm 0.02$ & $15.18 \pm 0.03$ & $14.35 \pm 0.15$ & $12.45 \pm 0.02$ & $11.71 \pm 0.02$ & $11.49 \pm 0.02$ & $11.35 \pm 0.02$ & $11.21 \pm 0.02$ & $11.35 \pm 0.19$ \\ 
201505350 & $13.80 \pm 0.02$ & $13.00 \pm 0.01$ & $13.36 \pm 0.02$ & $12.76 \pm 0.01$ & $12.57 \pm 0.02$ & $11.60 \pm 0.02$ & $11.21 \pm 0.02$ & $11.16 \pm 0.03$ & $11.10 \pm 0.02$ & $11.13 \pm 0.02$ & $10.95 \pm 0.12$ \\ 
201546283 & $13.51 \pm 0.07$ & $12.64 \pm 0.02$ & $13.03 \pm 0.02$ & $12.37 \pm 0.02$ & $12.17 \pm 0.05$ & $11.16 \pm 0.02$ & $10.79 \pm 0.03$ & $10.70 \pm 0.02$ & $10.61 \pm 0.02$ & $10.66 \pm 0.02$ & $10.53 \pm 0.09$ \\ 
201549860 & $15.56 \pm 0.06$ & $14.37 \pm 0.05$ & $14.95 \pm 0.07$ & $13.85 \pm 0.03$ & $13.45 \pm 0.05$ & $12.14 \pm 0.02$ & $11.56 \pm 0.02$ & $11.42 \pm 0.02$ & $11.38 \pm 0.02$ & $11.46 \pm 0.02$ & $11.60 \pm 0.25$ \\ 
201555883 & $16.48 \pm 0.01$ & $15.43 \pm 0.01$ & $16.19 \pm 0.10$ & $15.09 \pm 0.13$ & $14.55 \pm 0.08$ & $13.20 \pm 0.02$ & $12.53 \pm 0.03$ & $12.43 \pm 0.03$ & $12.34 \pm 0.02$ & $12.38 \pm 0.03$ & --- \\ 
201565013 & --- & --- & $18.25 \pm 0.01$ & $16.91 \pm 0.01$ & $16.34 \pm 0.01$ & $14.78 \pm 0.04$ & $14.11 \pm 0.05$ & $14.08 \pm 0.07$ & $13.94 \pm 0.03$ & $13.87 \pm 0.04$ & --- \\ 
201569483 & $12.90 \pm 0.08$ & $12.05 \pm 0.07$ & $12.44 \pm 0.03$ & $11.76 \pm 0.08$ & $11.48 \pm 0.08$ & $10.39 \pm 0.02$ & $9.97 \pm 0.03$ & $9.88 \pm 0.02$ & $9.82 \pm 0.02$ & $9.87 \pm 0.02$ & $9.82 \pm 0.05$ \\ 
201577035 & $13.14 \pm 0.11$ & $12.42 \pm 0.02$ & $12.70 \pm 0.04$ & $12.21 \pm 0.03$ & $12.13 \pm 0.20$ & $11.06 \pm 0.02$ & $10.75 \pm 0.02$ & $10.64 \pm 0.02$ & $10.64 \pm 0.02$ & $10.69 \pm 0.02$ & $10.55 \pm 0.10$ \\ 
201596316 & $14.21 \pm 0.01$ & $13.39 \pm 0.09$ & $13.78 \pm 0.07$ & $13.14 \pm 0.12$ & $12.88 \pm 0.10$ & $11.87 \pm 0.02$ & $11.46 \pm 0.02$ & $11.35 \pm 0.02$ & $11.29 \pm 0.02$ & $11.35 \pm 0.02$ & $10.80 \pm 0.11$ \\ 
201613023 & $12.99 \pm 0.09$ & $12.26 \pm 0.01$ & $12.56 \pm 0.03$ & $12.05 \pm 0.03$ & $11.96 \pm 0.08$ & $10.98 \pm 0.02$ & $10.71 \pm 0.02$ & $10.61 \pm 0.02$ & $10.58 \pm 0.02$ & $10.63 \pm 0.02$ & $10.59 \pm 0.10$ \\ 
201617985 & $16.34 \pm 0.02$ & $14.86 \pm 0.05$ & $15.62 \pm 0.06$ & $14.26 \pm 0.08$ & $13.42 \pm 0.09$ & $11.72 \pm 0.02$ & $11.09 \pm 0.04$ & $10.90 \pm 0.02$ & $10.73 \pm 0.02$ & $10.70 \pm 0.02$ & $10.86 \pm 0.11$ \\ 
201629650 & $13.61 \pm 0.03$ & $12.90 \pm 0.04$ & $13.20 \pm 0.03$ & $12.73 \pm 0.01$ & $12.53 \pm 0.06$ & $11.57 \pm 0.03$ & $11.26 \pm 0.02$ & $11.17 \pm 0.03$ & $11.14 \pm 0.02$ & $11.18 \pm 0.02$ & $10.93 \pm 0.12$ \\ 
201635569 & $17.74 \pm 0.16$ & $16.31 \pm 0.01$ & $17.02 \pm 0.01$ & $15.62 \pm 0.01$ & $14.87 \pm 0.01$ & $13.42 \pm 0.03$ & $12.77 \pm 0.02$ & $12.61 \pm 0.03$ & $12.52 \pm 0.03$ & $12.55 \pm 0.03$ & --- \\ 
201649426 & $14.57 \pm 0.03$ & $13.53 \pm 0.01$ & $14.04 \pm 0.01$ & $13.18 \pm 0.02$ & $12.86 \pm 0.06$ & $11.57 \pm 0.02$ & $11.07 \pm 0.02$ & $11.07 \pm 0.02$ & $10.88 \pm 0.02$ & $10.91 \pm 0.02$ & $10.86 \pm 0.12$ \\ 
201702477 & $15.27 \pm 0.05$ & $14.57 \pm 0.04$ & $14.89 \pm 0.04$ & $14.40 \pm 0.06$ & $14.24 \pm 0.03$ & $13.27 \pm 0.03$ & $12.88 \pm 0.03$ & $12.77 \pm 0.03$ & $12.81 \pm 0.02$ & $12.84 \pm 0.03$ & --- \\ 
201736247 & $15.49 \pm 0.06$ & $14.66 \pm 0.05$ & $15.01 \pm 0.04$ & $14.35 \pm 0.04$ & $14.14 \pm 0.02$ & $13.07 \pm 0.02$ & $12.55 \pm 0.02$ & $12.49 \pm 0.03$ & $12.46 \pm 0.02$ & $12.50 \pm 0.02$ & --- \\ 
201754305 & $15.65 \pm 0.04$ & $14.65 \pm 0.01$ & $15.13 \pm 0.04$ & $14.28 \pm 0.01$ & $13.93 \pm 0.05$ & $12.76 \pm 0.03$ & $12.21 \pm 0.03$ & $12.09 \pm 0.02$ & $12.06 \pm 0.02$ & $12.10 \pm 0.02$ & $12.34 \pm 0.46$ \\ 
201779067 & $11.81 \pm 0.01$ & $11.27 \pm 0.01$ & $11.53 \pm 0.07$ & $11.12 \pm 0.01$ & $10.95 \pm 0.01$ & $10.13 \pm 0.02$ & $9.87 \pm 0.02$ & $9.80 \pm 0.02$ & $9.74 \pm 0.02$ & $9.77 \pm 0.02$ & $9.74 \pm 0.04$ \\ 
201828749 & $12.48 \pm 0.04$ & $11.76 \pm 0.01$ & $12.13 \pm 0.05$ & $11.58 \pm 0.04$ & $11.32 \pm 0.04$ & $10.49 \pm 0.03$ & $10.23 \pm 0.04$ & $9.93 \pm 0.03$ & $9.82 \pm 0.02$ & $9.87 \pm 0.02$ & $9.98 \pm 0.06$ \\ 
201855371 & $14.82 \pm 0.06$ & $13.52 \pm 0.04$ & $14.20 \pm 0.06$ & $12.96 \pm 0.03$ & $12.45 \pm 0.01$ & $11.08 \pm 0.02$ & $10.44 \pm 0.02$ & $10.31 \pm 0.02$ & $10.22 \pm 0.02$ & $10.26 \pm 0.02$ & $10.12 \pm 0.07$ \\ 
201912552 & $15.01 \pm 0.06$ & $13.50 \pm 0.05$ & $14.22 \pm 0.05$ & $12.86 \pm 0.04$ & $11.66 \pm 0.08$ & $9.76 \pm 0.03$ & $9.13 \pm 0.03$ & $8.90 \pm 0.02$ & $8.77 \pm 0.02$ & $8.67 \pm 0.02$ & $8.55 \pm 0.03$ \\ 
201929294 & $14.32 \pm 0.04$ & $13.31 \pm 0.03$ & $13.78 \pm 0.05$ & $12.97 \pm 0.07$ & $12.61 \pm 0.09$ & $11.48 \pm 0.03$ & $10.98 \pm 0.02$ & $10.80 \pm 0.02$ & $10.73 \pm 0.02$ & $10.78 \pm 0.02$ & $10.67 \pm 0.10$ 
\enddata
%\tablecomments{}
%\tablenotetext{$\dagger$}{previously undetected TTV signal.}
\end{deluxetable*}



\clearpage
%\LongTables
\begin{deluxetable*}{cccccccc}
\tablewidth{0pt}
\tabletypesize{\scriptsize}
\tablecaption{Stellar Properties for all Objects of Interest}
\tablehead{
\colhead{EPIC} &
\colhead{RA (J2000)} &
\colhead{Dec (J2000)} &
\colhead{Mass ($M_\odot$)} &
\colhead{Radius ($R_\odot$)} &
\colhead{\teff} &
\colhead{[Fe/H]} &
\colhead{Distance} 
}
\label{Tab:Stars}
\startdata
201208431 & 174.74564 & -3.905585 &  & & & & \\
201257461 & 178.161109 & -3.094936 &  & & & & \\
201295312 & 174.01163 & -2.520881 &  & & & & \\
201338508 & 169.303502 & -1.877976 &  & & & & \\
201367065 & 172.334949 & -1.454787 &  & & & & \\
201384232 & 178.19226 & -1.198477 &  & & & & \\
201393098 & 167.093771 & -1.065755 &  & & & & \\
201403446 & 174.266344 & -0.907261 &  & & & & \\
201445392 & 169.793665 & -0.284375 &  & & & & \\
201465501 & 176.264468 & 0.005301 &  & & & & \\
201505350 & 174.960319 & 0.603575 &  & & & & \\
201546283 & 171.515165 & 1.230738 &  & & & & \\
201549860 & 170.103081 & 1.285956 &  & & & & \\
201555883 & 176.07594 & 1.375947 &  & & & & \\
201565013 & 176.992193 & 1.510249 &  & & & & \\
201569483 & 167.171299 & 1.577513 &  & & & & \\
201577035 & 172.121957 & 1.690636 &  & & & & \\
201596316 & 169.042002 & 1.98684 &  & & & & \\
201613023 & 173.192036 & 2.244884 &  & & & & \\
201617985 & 179.491659 & 2.321476 &  & & & & \\
201629650 & 170.155528 & 2.502696 &  & & & & \\
201635569 & 178.057026 & 2.594245 &  & & & & \\
201649426 & 177.234262 & 2.807619 &  & & & & \\
201702477 & 175.240794 & 3.681584 &  & & & & \\
201736247 & 178.110797 & 4.254747 &  & & & & \\
201754305 & 175.097258 & 4.55734 &  & & & & \\
201779067 & 168.542699 & 4.988131 &  & & & & \\
201828749 & 175.654342 & 5.894323 &  & & & & \\
201855371 & 178.329775 & 6.412261 &  & & & & \\
201912552 & 172.56046 & 7.588391  & $0.413 \pm 0.043$ & $0.394 \pm 0.038$ & $3503 \pm 60$ & $0.09 \pm 0.09$ & \\
201929294 & 174.656969 & 7.959611 &  & & & & 
\enddata
%\tablecomments{}
%\tablenotetext{$\dagger$}{previously undetected TTV signal.}
\end{deluxetable*}


\end{document}

